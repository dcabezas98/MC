\documentclass[12pt,spanish]{article}
% aprovechamiento de la p\'agina -- fill an A4 (210mm x 297mm) page
% Note: 1 inch = 25.4 mm = 72.27 pt
% 1 pt = 3.5 mm (approx)

% vertical page layout -- one inch margin top and bottom
\topmargin      -10 mm   % top margin less 1 inch
\headheight       0 mm   % height of box containing the head
\headsep          0 mm   % space between the head and the body of the page
\textheight     255 mm   % the height of text on the page
\footskip         7 mm   % distance from bottom of body to bottom of foot

% horizontal page layout -- one inch margin each side
\oddsidemargin    0 mm     % inner margin less one inch on odd pages
\evensidemargin   0 mm     % inner margin less one inch on even pages
\textwidth      159 mm     % normal width of text on page

\setlength{\parindent}{0pt}

\usepackage{tikz}
\usetikzlibrary{automata,positioning}

\usepackage[doument]{ragged2e}
\usepackage{babel}
\usepackage[utf8]{inputenc}
\usepackage{amsmath,amsthm,mathtools}
\usepackage{amsfonts,amssymb,latexsym}
\usepackage{enumerate}
\usepackage{subfigure, float, graphicx, caption}
\captionsetup[table]{labelformat=empty}
\captionsetup[figure]{labelformat=empty}
\definecolor{RojoAnayelRey}{rgb}{1,.25,.25}
\usepackage[bookmarks=true,
            bookmarksnumbered=false, % true means bookmarks in 
                                     % left window are numbered                         
            bookmarksopen=false,     % true means only level 1
                                     % are displayed.
            colorlinks=true,
            linkcolor=webred]{hyperref}
\definecolor{webgreen}{rgb}{0, 0.5, 0} % less intense green
\definecolor{webblue}{rgb}{0, 0, 0.5}  % less intense blue
\definecolor{webred}{rgb}{0.5, 0, 0}   % less intense red
\definecolor{dkgreen}{rgb}{0,0.6,0}
\definecolor{gray}{rgb}{0.5,0.5,0.5}
\definecolor{mauve}{rgb}{0.58,0,0.82}
\definecolor{MistyRose}{RGB}{255,228,225}
\definecolor{LightCyan}{RGB}{224,255,255}

% \usepackage{beton}
% \usepackage[T1]{fontenc}

% Theorem environments

%% \theoremstyle{plain} %% This is the default
\newtheorem{theorem}{Teorema}[section]
\newtheorem{corollary}[theorem]{Corolario}
\newtheorem{lemma}[theorem]{Lema}
\newtheorem{proposition}[theorem]{Proposici\'on}
%\newtheorem{ax}{Axioma}

\theoremstyle{definition}
\newtheorem{definition}{Definici\'on}[section]
\newtheorem{algorithm}{\textrm{\bf Algoritmo}}[section]

\theoremstyle{remark}
\newtheorem{remark}{Observaci\'on}[section]
\newtheorem{example}{Ejemplo}[section]
\newtheorem{exercise}{Ejercicio}%[section]
%\newenvironment{solution}{\begin{proof}[Solution]}{\end{proof}}
\newenvironment{solution}{\begin{proof}[Solución]}{\end{proof}}
\newtheorem*{notation}{Notaci\'on}

%\numberwithin{equation}{section}

%\newcommand{\regla}[2]{
%\begin{array}{c}
%#1\\
%\hline
%#2\\
%\end{array}
%}
\begin{document}

\title{Modelos de Computación: \\ Relación de problemas 1}
\author{David Cabezas Berrido}
\date{\vspace{-5mm}}
\maketitle

\setcounter{exercise}{16}
\begin{exercise}~ Autómata finito determinista que reconoce el
  lenguaje
  \[L_1=\{u\in\{0,1\}^* \ | \text{ el número de 1's no es múltiplo de 3}\}\] \vspace{-10mm}
  \begin{figure}[H]
  \centering
  \subfigure{\includegraphics[width=60mm]{17_1}}
\end{figure}
\vspace{-10mm}
  \[L_2=\{u\in\{0,1\}^* \ |\text{ el número de 0's es par}\}\] \vspace{-10mm}
  \begin{figure}[H]
  \centering
  \subfigure{\includegraphics[width=70mm]{17_2}}
  \end{figure} \vspace{-5mm}
  Ahora sólo tenemos que intersecar los dos autómatas para formar el
  deseado, el que reconoce el lenguaje
  \[L_3=\{u\in\{0,1\}^* \ |\text{ el número de 1's no es múltiplo de 3 y el número de 0's es par}\}\]
  \begin{figure}[H]
  \centering
  \subfigure{\includegraphics[width=75mm]{17_3}}
  \end{figure}
  
\end{exercise}

\setcounter{exercise}{21}
\begin{exercise}~ Para hallar la expresión regular que representa el lenguaje aceptado por el autómata \vspace{-5mm}
\begin{figure}[H]
  \centering
  \subfigure{\includegraphics[width=50mm]{22}}
\end{figure} \vspace{-5mm}
usaremos el algoritmo visto en clase (sólo mostraré un par de iteraciones)
\begin{align*}
  r_{11}^3&+r_{13}^3 \\
  r_{11}^2+r_{13}^2(r_{33}^2)^*r_{31}^2&+r_{13}^2+r_{13}^2(r_{33}^2)^*r_{33}^2 \\
  \varepsilon+(a+b)a^*b(a(a+b)a^*b)^*a&+(a+b)a^*b(a(a+b)a^*b)^* \\
  \varepsilon+(a+b)a^*b(a(a&+b)a^*b)^*(a+\varepsilon)
\end{align*}
\end{exercise}

\begin{exercise}~ Para probar que
  $B_n=\{a^k \ | \text{ k es múltiplo de n}\}=\{a^{kn} \ | \ k \in
  \mathbb{N}\}$ es regular para todo $n$, construiremos un autómata
  finito determinista que lo reconozca.

  $B_0=\{\varepsilon\}$ es trivialmente regular, lo reconoce el autómata \vspace{-5mm}
\begin{figure}[H]
  \centering
  \subfigure{\includegraphics[width=50mm]{23_b0}}
\end{figure} \vspace{-5mm}
$B_1=\{a\}^*$ es reconocido por \vspace{-7mm}
\begin{figure}[H]
  \centering
  \subfigure{\includegraphics[width=20mm]{23_b1}}
\end{figure}
$B_2=$ palabras sobre $\{a\}^*$ con número par de $a$'s es reconocido por
\begin{figure}[H]
  \centering
  \subfigure{\includegraphics[width=50mm]{23_b2}}
\end{figure}

$B_3$ por 
\begin{figure}[H]
  \centering
  \subfigure{\includegraphics[width=50mm]{23_b3}}
\end{figure}

De esta forma,
$M_n=(\{q_0,\ldots,q_{n-1}\},\{a\},q_0,\delta_n,\{q_0\})$ con
\\ $\delta_n(q_i,a)=q_{(i+1)\%n}$ $\forall i=0,\ldots,n-1$ es un
autómata finito determinista que reconoce el lenguaje $B_n$
$\forall n\geq 2$.

\end{exercise}

\end{document}
