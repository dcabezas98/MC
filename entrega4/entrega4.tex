\documentclass[12pt,spanish]{article}
% aprovechamiento de la p\'agina -- fill an A4 (210mm x 297mm) page
% Note: 1 inch = 25.4 mm = 72.27 pt
% 1 pt = 3.5 mm (approx)

% vertical page layout -- one inch margin top and bottom
\topmargin      -10 mm   % top margin less 1 inch
\headheight       0 mm   % height of box containing the head
\headsep          0 mm   % space between the head and the body of the page
\textheight     255 mm   % the height of text on the page
\footskip         7 mm   % distance from bottom of body to bottom of foot

% horizontal page layout -- one inch margin each side
\oddsidemargin    0 mm     % inner margin less one inch on odd pages
\evensidemargin   0 mm     % inner margin less one inch on even pages
\textwidth      159 mm     % normal width of text on page

\setlength{\parindent}{0pt}

\usepackage{tikz}
\usetikzlibrary{automata,positioning}

\usepackage[doument]{ragged2e}
\usepackage{babel}
\usepackage[utf8]{inputenc}
\usepackage{amsmath,amsthm,mathtools}
\usepackage{amsfonts,amssymb,latexsym}
\usepackage{enumerate}
\usepackage{subfigure, float, graphicx, caption}
\captionsetup[table]{labelformat=empty}
\captionsetup[figure]{labelformat=empty}
\definecolor{RojoAnayelRey}{rgb}{1,.25,.25}
\usepackage[bookmarks=true,
            bookmarksnumbered=false, % true means bookmarks in 
                                     % left window are numbered         
            bookmarksopen=false,     % true means only level 1
                                     % are displayed.
            colorlinks=true,
            linkcolor=webred]{hyperref}
\definecolor{webgreen}{rgb}{0, 0.5, 0} % less intense green
\definecolor{webblue}{rgb}{0, 0, 0.5}  % less intense blue
\definecolor{webred}{rgb}{0.5, 0, 0}   % less intense red
\definecolor{dkgreen}{rgb}{0,0.6,0}
\definecolor{gray}{rgb}{0.5,0.5,0.5}
\definecolor{mauve}{rgb}{0.58,0,0.82}
\definecolor{MistyRose}{RGB}{255,228,225}
\definecolor{LightCyan}{RGB}{224,255,255}

% \usepackage{beton}
% \usepackage[T1]{fontenc}

% Theorem environments

%% \theoremstyle{plain} %% This is the default
\newtheorem{theorem}{Teorema}[section]
\newtheorem{corollary}[theorem]{Corolario}
\newtheorem{lemma}[theorem]{Lema}
\newtheorem{proposition}[theorem]{Proposici\'on}
%\newtheorem{ax}{Axioma}

\theoremstyle{definition}
\newtheorem{definition}{Definici\'on}[section]
\newtheorem{algorithm}{\textrm{\bf Algoritmo}}[section]

\theoremstyle{remark}
\newtheorem{remark}{Observaci\'on}[section]
\newtheorem{example}{Ejemplo}[section]
\newtheorem{exercise}{\textbf{Ejercicio}}%[section]
%\newenvironment{solution}{\begin{proof}[Solution]}{\end{proof}}
\newenvironment{solution}{\begin{proof}[Solución]}{\end{proof}}
\newtheorem*{notation}{Notaci\'on}

%\numberwithin{equation}{section}

%\newcommand{\regla}[2]{
%\begin{array}{c}
%#1\\
%\hline
%#2\\
%\end{array}
%}
\begin{document}

\title{Modelos de Computación: \\ Relación de problemas 4}
\author{David Cabezas Berrido}
\date{\vspace{-5mm}}
\maketitle

\setcounter{exercise}{13}

\begin{exercise}~ Dar gramáticas idependientes del contexto que
  generen los siguientes lenguajes sobre el alfabeto $A=\{0,1\}$.

  \begin{enumerate}[a)]
  \item $L_1$: palabras que si empiezan por 0, tienen el mismo número
    de 0s que de 1s.

    \[G_1=(\{S,D,B,C\},A,P,S)\]
    Donde \vspace{-5mm}
    \begin{align*}
      P=\{S&\rightarrow 0D|1B|\varepsilon, \\
      D&\rightarrow C1C, \\
      C&\rightarrow C0C1C|C1C0C|\varepsilon, \\
      B&\rightarrow 0B|1B|\varepsilon\}
    \end{align*}
    
  \item $L_2$: palabras que si terminan por 1, tienen un número de 1s
    mayor o igual que el número de 0s.

    \[G_2=(\{S,X,Y,Z\},A,P,S)\]
    Donde \vspace{-5mm}
    \begin{align*}
      P=\{S&\rightarrow X1|Y0|\varepsilon, \\
      X&\rightarrow Z0Z|Z, \\
      Z&\rightarrow Z1Z|Z0Z1Z|Z1Z0Z|\varepsilon, \\
      Y&\rightarrow 0Y|1Y|\varepsilon\}
    \end{align*}

  \item $L_1\cap L_2$.
    \[G_3=(\{S,B,C,D,X,Y,Z\},A,P,S)\]
    Donde \vspace{-5mm}
    \begin{align*}
      P=\{S&\rightarrow 0C|1X|\varepsilon, \\
      C&\rightarrow D1D, \\
      D&\rightarrow D0D1D|D1D0D|\varepsilon, \\
      X&\rightarrow Y1|B0|\varepsilon, \\
      Y&\rightarrow Z0Z0Z|Z0Z|Z, \\
      Z&\rightarrow Z1Z|Z0Z1Z|Z1Z0Z|\varepsilon, \\
      B&\rightarrow 0B|1B|\varepsilon\}
    \end{align*}
  \end{enumerate}
  
\end{exercise}

\newpage

\setcounter{exercise}{15}

\begin{exercise}~ Una gramática independiente del contexto
  generalizada es una gramática en la que las producciones son de la
  forma $A\rightarrow \textbf{r}$ donde $\textbf{r}$ es una expresión
  regular de variables y símbolos terminales. Demostrar que un
  lenguaje es independiente del contexto si y sólo si se puede generar
  por una gramática generalizada. \\

  ($\Longrightarrow$) Si $L$ es un lenguaje independiente del contexto,
  existe una gramática independiente del contexto que lo genera.
  \[G=(V,T,P,S)\] donde todas las reglas son de la forma
  $A\rightarrow \alpha$ con $\alpha \in (V \cup T)^*$. Estos $\alpha$
  son expresiones regulares (concatenaciones de variables y símbolos),
  luego $G$ es una gramática independiente del contexto generalizada. \\

  ($\Longleftarrow$) Sea $L$ es un lenguaje generado por una gramática
  independiente del contexto generalizada.
  \[G=(V,T,P,S)\] donde todas las reglas son de la forma
  $A\rightarrow \textbf{r}$ con $\textbf{r}$ una expresión regular de
  variables y símbolos terminales. Para probar que existe una
  gramática de tipo 2 que genera $L$, debo probar que cada regla
  $A\rightarrow \textbf{r}$ puede sustituirse por un número finito de
  reglas de producción de la forma $A\rightarrow \alpha$ con
  $\alpha \in (V\cup T)^*$. Haremos el mismo razonamiento que el que
  se hizo en teoría para probar que el lenguaje asociado a una
  expresión regular puede expresarse con un autómata, construir el
  conjunto finito de reglas de producción de forma recursiva.

  \begin{itemize}
  \item $A\rightarrow\emptyset$ puede eliminarse.
  \item $A\rightarrow\alpha$ \quad ($\alpha \in (V \cup T)^*$) ya está en la
    forma deseada.
  \item $A\rightarrow (r_1+r_2)$ puede sustituirse por dos reglas:
    $A\rightarrow r_1$, $A\rightarrow r_2$.
  \item $A\rightarrow (r_1r_2)$ puede sustituirse por las reglas:
    $A\rightarrow BC$, $B\rightarrow r_1$, $C\rightarrow r_2$.
  \item $A\rightarrow r^*$ puede sustituirse por:
    $B\rightarrow AB|\varepsilon$, $A\rightarrow r$.
  \end{itemize}

  Hemos probado que cada regla $A\rightarrow \textbf{r}$ puede
  sustituirse por un número finito de reglas $A\rightarrow
  \alpha$. Luego toda gramática libre de contexto generalizada tiene
  una gramática libre de contexto equivalente.
\end{exercise}

~

\begin{exercise}~ Demostrar que los siguientes lenguajes son
  independientes del contexto:

  \begin{enumerate}[a)]
  \item $L_1=\{u \# w|u^{-1} \text{ es una subcadena de $w$; } u,w \in \{0,1\}^*\}$
    
    Es generado por la gramática libre de
    contexto \[G=(\{S,A,B\},\{0,1,\#\},P,S)\] Donde \vspace{-5mm}
    \begin{align*}
      P=\{S&\rightarrow S0|S1|A, \\
      A&\rightarrow 0A0|1A1|B, \\
      B&\rightarrow B0|B1|\#\}
    \end{align*}

  \item $L_2=\{u_1\# u_2\#\ldots\# u_k \ |\ k\geq 1, \text{ cada } u_i \in \{0,1\}^*, \text{ y para algún } i \text{ y } j,\ u_i = u_j^{-1}\}$

    Es generado por la gramática libre de
    contexto \[G=(\{S,A,C\},\{0,1,\#\},P,S)\] Donde \vspace{-5mm}
    \begin{align*}
      P=\{S&\rightarrow \varepsilon|A|C\# A\# C|A\# C|C\# A, \\
      A&\rightarrow 0A0|1A1|\# C\#|\#, \\
      C&\rightarrow 0C|1C|C\# C| \varepsilon\}
    \end{align*}
  \end{enumerate}
\end{exercise}

\setcounter{exercise}{18}

~

\begin{exercise}~ Probar que
  $L=\{u\# v \ |\ u,v \in \{0,1\}^*,\ u\neq v\}$ es un lenguaje libre
  de contexto. \\

  Debo encontrar una gramática libre de contexto que lo genere.
  \[G=(\{S,A,X,Y,C,H,T,H_0,H_1,K\},\{0,1,\#\},P,S)\]
  \vspace{-10mm}
  \begin{align*}
    P=\{S&\rightarrow A|C, \\
    A&\rightarrow KAK|XK|KY, \hspace{10mm}   (|u|\neq |v|) \\
    X&\rightarrow XK|\#,     \hspace{24.9mm} (|u| < |v|) \\
    Y&\rightarrow KY|\#,     \hspace{25.4mm} (|u| > |v|) \\
    C&\rightarrow HT,        \hspace{30.6mm} (|u|=|v|) \\
    T&\rightarrow KT|\varepsilon \\
    H&\rightarrow H_00|H_11 \\
    H_0&\rightarrow KH_0K|1T\# \\
    H_1&\rightarrow KH_1K|0T\# \\
    K&\rightarrow 0|1\}
  \end{align*}
\end{exercise}

\setcounter{exercise}{20}

~

\begin{exercise}~ Dar gramáticas independientes del contexto no
  ambiguas para los siguientes lenguajes sobre el alfabeto $\{0,1\}$:

  \begin{enumerate}[a)]
  \item Palabras $w$ tales que en todo prefijo de $w$, el número de 0s
    es mayor o igual que el de 1s.
    \[G=(\{S,X,Y\},\{0,1\},P,S)\]
    Donde \vspace{-5mm}
    \begin{align*}
      P=\{S&\rightarrow S0|SX|\varepsilon \\
      X&\rightarrow Y1 \\
      Y&\rightarrow YY1|0\}
    \end{align*}

  \item Palabras $w$ en las que el número de 0s es mayor o igual que
    el número de 1s.
    \[G=(\{S,A,B,C,D,H,T\},\{0,1\},P,S)\]
    Donde \vspace{-5mm}
    \begin{align*}
      P=\{S&\rightarrow H|T \\
      H&\rightarrow AH|CH|\varepsilon \\
      T&\rightarrow 0S|CT \\
      A&\rightarrow 0B \\
      B&\rightarrow 0BB|1 \\
      C&\rightarrow 1D \\
      D&\rightarrow 1DD|0 \}
    \end{align*}
  \end{enumerate}
  
\end{exercise}

\end{document}
