\documentclass[12pt,spanish]{article}
% aprovechamiento de la p\'agina -- fill an A4 (210mm x 297mm) page
% Note: 1 inch = 25.4 mm = 72.27 pt
% 1 pt = 3.5 mm (approx)

% vertical page layout -- one inch margin top and bottom
\topmargin      -10 mm   % top margin less 1 inch
\headheight       0 mm   % height of box containing the head
\headsep          0 mm   % space between the head and the body of the page
\textheight     255 mm   % the height of text on the page
\footskip         7 mm   % distance from bottom of body to bottom of foot

% horizontal page layout -- one inch margin each side
\oddsidemargin    0 mm     % inner margin less one inch on odd pages
\evensidemargin   0 mm     % inner margin less one inch on even pages
\textwidth      159 mm     % normal width of text on page

\setlength{\parindent}{0pt}

\usepackage{tikz}
\usetikzlibrary{automata,positioning}

\usepackage[doument]{ragged2e}
\usepackage{babel}
\usepackage[utf8]{inputenc}
\usepackage{amsmath,amsthm,mathtools}
\usepackage{amsfonts,amssymb,latexsym}
\usepackage{enumerate}
\usepackage{subfigure, float, graphicx, caption}
\captionsetup[table]{labelformat=empty}
\captionsetup[figure]{labelformat=empty}
\definecolor{RojoAnayelRey}{rgb}{1,.25,.25}
\usepackage[bookmarks=true,
            bookmarksnumbered=false, % true means bookmarks in 
                                     % left window are numbered         
            bookmarksopen=false,     % true means only level 1
                                     % are displayed.
            colorlinks=true,
            linkcolor=webred]{hyperref}
\definecolor{webgreen}{rgb}{0, 0.5, 0} % less intense green
\definecolor{webblue}{rgb}{0, 0, 0.5}  % less intense blue
\definecolor{webred}{rgb}{0.5, 0, 0}   % less intense red
\definecolor{dkgreen}{rgb}{0,0.6,0}
\definecolor{gray}{rgb}{0.5,0.5,0.5}
\definecolor{mauve}{rgb}{0.58,0,0.82}
\definecolor{MistyRose}{RGB}{255,228,225}
\definecolor{LightCyan}{RGB}{224,255,255}

% \usepackage{beton}
% \usepackage[T1]{fontenc}

% Theorem environments

%% \theoremstyle{plain} %% This is the default
\newtheorem{theorem}{Teorema}[section]
\newtheorem{corollary}[theorem]{Corolario}
\newtheorem{lemma}[theorem]{Lema}
\newtheorem{proposition}[theorem]{Proposici\'on}
%\newtheorem{ax}{Axioma}

\theoremstyle{definition}
\newtheorem{definition}{Definici\'on}[section]
\newtheorem{algorithm}{\textrm{\bf Algoritmo}}[section]

\theoremstyle{remark}
\newtheorem{remark}{Observaci\'on}[section]
\newtheorem{example}{Ejemplo}[section]
\newtheorem{exercise}{\textbf{Ejercicio}}%[section]
%\newenvironment{solution}{\begin{proof}[Solution]}{\end{proof}}
\newenvironment{solution}{\begin{proof}[Solución]}{\end{proof}}
\newtheorem*{notation}{Notaci\'on}

%\numberwithin{equation}{section}

%\newcommand{\regla}[2]{
%\begin{array}{c}
%#1\\
%\hline
%#2\\
%\end{array}
%}
\begin{document}

\title{Modelos de Computación: \\ Relación de problemas 4}
\author{David Cabezas Berrido}
\date{\vspace{-5mm}}
\maketitle

\setcounter{exercise}{13}

\begin{exercise}~ Dar gramáticas idependientes del contexto que
  generen los siguientes lenguajes sobre el alfabeto $A=\{0,1\}$.

  \begin{enumerate}[a)]
  \item $L_1$: palabras que si empiezan por 0, tienen el mismo número
    de 0s que de 1s.

    \[G_1=(\{S,D,B,C\},A,P,S)\]
    Donde \vspace{-5mm}
    \begin{align*}
      P=\{S&\rightarrow 0D|1B|\varepsilon, \\
      D&\rightarrow C1C, \\
      C&\rightarrow C0C1C|C1C0C|\varepsilon, \\
      B&\rightarrow 0B|1B|\varepsilon\}
    \end{align*}
    
  \item $L_2$: palabras que si terminan por 1, tienen un número de 1s
    mayor o igual que el número de 0s.

    \[G_2=(\{S,X,Y,Z\},A,P,S)\]
    Donde \vspace{-5mm}
    \begin{align*}
      P=\{S&\rightarrow X1|Y0|\varepsilon, \\
      X&\rightarrow Z0Z|Z, \\
      Z&\rightarrow Z1Z|Z0Z1Z|Z1Z0Z|\varepsilon, \\
      Y&\rightarrow 0Y|1Y|\varepsilon\}
    \end{align*}

  \item $L_1\cap L_2$.
    \[G_2=(\{S,B,C,D,X,Y,Z\},A,P,S)\]
    Donde \vspace{-5mm}
    \begin{align*}
      P=\{S&\rightarrow 0C|1X|\varepsilon, \\
      C&\rightarrow D1D, \\
      D&\rightarrow D0D1D|D1D0D|\varepsilon, \\
      X&\rightarrow Y1|B0, \\
      Y&\rightarrow Z0Z0Z|Z0Z|Z, \\
      Z&\rightarrow Z1Z|Z0Z1Z|Z1Z0Z|\varepsilon, \\
      B&\rightarrow 0B|1B|\varepsilon\}
    \end{align*}
  \end{enumerate}
  
\end{exercise}

\end{document}
