\documentclass[12pt,spanish]{article}

% aprovechamiento de la p\'agina -- fill an A4 (210mm x 297mm) page
% Note: 1 inch = 25.4 mm = 72.27 pt
% 1 pt = 3.5 mm (approx)

% vertical page layout -- one inch margin top and bottom
\topmargin      -10 mm   % top margin less 1 inch
\headheight       0 mm   % height of box containing the head
\headsep          0 mm   % space between the head and the body of the page
\textheight     255 mm   % the height of text on the page
\footskip         7 mm   % distance from bottom of body to bottom of foot

% horizontal page layout -- one inch margin each side
\oddsidemargin    0 mm     % inner margin less one inch on odd pages
\evensidemargin   0 mm     % inner margin less one inch on even pages
\textwidth      159 mm     % normal width of text on page

\setlength{\parindent}{0pt}

\usepackage[doument]{ragged2e}
\usepackage{babel}
\usepackage[utf8]{inputenc}
\usepackage{amsmath,amsthm,mathtools}
\usepackage{amsfonts,amssymb,latexsym}
\usepackage{enumerate}
\usepackage[dvips,usenames]{color}
\definecolor{RojoAnayelRey}{rgb}{1,.25,.25}
\usepackage{tikz}
\usepackage[bookmarks=true,
            bookmarksnumbered=false, % true means bookmarks in 
                                     % left window are numbered                         
            bookmarksopen=false,     % true means only level 1
                                     % are displayed.
            colorlinks=true,
            linkcolor=webred]{hyperref}
\definecolor{webgreen}{rgb}{0, 0.5, 0} % less intense green
\definecolor{webblue}{rgb}{0, 0, 0.5}  % less intense blue
\definecolor{webred}{rgb}{0.5, 0, 0}   % less intense red
\definecolor{dkgreen}{rgb}{0,0.6,0}
\definecolor{gray}{rgb}{0.5,0.5,0.5}
\definecolor{mauve}{rgb}{0.58,0,0.82}
\definecolor{MistyRose}{RGB}{255,228,225}
\definecolor{LightCyan}{RGB}{224,255,255}

% \usepackage{beton}
% \usepackage[T1]{fontenc}

% Theorem environments

%% \theoremstyle{plain} %% This is the default
\newtheorem{theorem}{Teorema}[section]
\newtheorem{corollary}[theorem]{Corolario}
\newtheorem{lemma}[theorem]{Lema}
\newtheorem{proposition}[theorem]{Proposici\'on}
%\newtheorem{ax}{Axioma}

\theoremstyle{definition}
\newtheorem{definition}{Definici\'on}[section]
\newtheorem{algorithm}{\textrm{\bf Algoritmo}}[section]

\theoremstyle{remark}
\newtheorem{remark}{Observaci\'on}[section]
\newtheorem{example}{Ejemplo}[section]
\newtheorem{exercise}{Ejercicio}%[section]
%\newenvironment{solution}{\begin{proof}[Solution]}{\end{proof}}
\newenvironment{solution}{\begin{proof}[Solución]}{\end{proof}}
\newtheorem*{notation}{Notaci\'on}

%\numberwithin{equation}{section}

%\newcommand{\regla}[2]{
%\begin{array}{c}
%#1\\
%\hline
%#2\\
%\end{array}
%}
\begin{document}

\title{Modelos de Computación: \\ Relación de problemas 1}
\author{David Cabezas Berrido}
\date{\vspace{-5mm}}
\maketitle

\setcounter{exercise}{12}
\begin{exercise}~ Sí que existe un procedimitendo algorítmico para
  comprobar si dos homomorfismos $f,g:A^*\rightarrow B^*$ son iguales,
  puesto que basta comprobar que $f|_A = g|_A$. Para comprobar esto
  sabemos que existe un algoritmo porque $A$ es finito, obviamente
  esta condición es necesaria, probemos que es suficiente
  ($f|_A = g|_A \implies f=g$). \\

Tomemos $x \in A^*$, será de la forma $x=a_1a_2\ldots a_m$ para algún
$m\geq 0$ y con $a_i \in A \ \forall i=1,\ldots,m$.  Usando la
condición y que $f$ y $g$ son homomorfismos tenemos
\[f(x)=f(a_1a_2\ldots a_m)=f(a_1)f(a_2)\ldots
  f(a_m)=g(a_1)g(a_2)\ldots g(a_m)=g(a_1a_2\ldots a_m)=g(x)\]
como queríamos.
\end{exercise}
~
\setcounter{exercise}{15}
\begin{exercise}~ La gramática $G=(\{S,A\},\{a,b\},P,S)$ donde \\
  $P=\{S\rightarrow abAS, abA\rightarrow baab, S \rightarrow a, A
  \rightarrow b\}$ genera el lenguaje
  \[L=\{u_1u_2\ldots u_na: n\geq 0, u_i \in \{baab,abb\} \ \forall
    i=1,\ldots,n\}\]

  La idea es que una $S$, siempre produce a la derecha otra $S$ o una
  $a$, luego cualquier palabra generada debe terminar en $a$. Para
  generar las secuencias $baab$ y $abb$ a la izquierda de $S$ se
  deriva $S \implies abAS \implies baabS$ ó
  $S \implies abAS \implies abbS$. Como no hay más reglas de
  derivación, estas son las únicas secuencias que se pueden generar.
\end{exercise}
~
\begin{exercise}~ La gramática $G=(V,T,P,S)$ donde:
  \begin{itemize}
  \item $V=\{<numero>,<digito>\}$
  \item $T=\{0,1,2,3,4,5,6,7,8,9\}$
  \item $S=<numero>$
  \item $P=\{<numero> \rightarrow <numero><digito>, \\
    <numero>\rightarrow <digito>, \\ <digito> \rightarrow
    0|1|2|3|4|5|6|7|8|9\}$
  \end{itemize}

  genera el lenguaje de los números naturales. Es claro que cualquier
  palabra generada va a ser una secuencia de dígitos correspondiente a
  cualquier número natural. Además podemos generar cualquier secuencia
  de dígitos de derecha a izquierda, luego cualquier número natural es
  una palabra del lenguaje generado.
\end{exercise}
~
\begin{exercise}~ La gramática $G=(\{A,S\},\{a,b\},P,S)$ donde \\
  $P=\{S\rightarrow aS, S \rightarrow aA, A \rightarrow bA, A
  \rightarrow b\}$ genera el lenguaje $L=\{a^nb^m: n,m \geq 1\}$.
  Partiendo desde $S$ puedo generar a la izquierda todas las $a$ que
  quiera hasta generar una $A$ a la derecha (se genera una $a$ como
  mínimo), desde $A$ puedo generar todas las $b$ que quiera a la
  izquierda (a continuación de las $a$) hasta finalizar (por lo menos
  se genera una $b$).
\end{exercise}
~
\begin{exercise}~ Sobre alfabeto $\{a,b,c\}$, una gramática lineal por
  la derecha o independiente del contexto que genere el lenguaje $L$
  será $G=(\{S,X,Y\},\{a,b,c\},P,S)$ donde
  \begin{itemize}
  \item $L$ es el lenguaje de las palabras que no contienen dos
    símbolos $b$ consecutivos:
    $P=\{S\rightarrow aS|bX|cS|\varepsilon, \ X\rightarrow
    aS|cS|\varepsilon\}$.
  \item $L$ es el lenguaje de las palabras que contienen dos símbolos
    $b$ consecutivos: \\
    $P=\{S\rightarrow aS|bX|cS, \ X\rightarrow aS|bY|cS, \ Y
    \rightarrow aY|bY|cY|\varepsilon\}$
  \item $L$ es el lenguaje de las palabras que contienen un número
    impar de símbolos $c$:
    $P=\{S\rightarrow aS|bS|cX, \ X\rightarrow aX|bX|cS|\varepsilon\}$
  \item $L$ es el lenguaje de las palabras que no contienen el mismo
    número de símbolos $b$ que de símbolos $c$:

    Definiré dos gramáticas, la que genera las palabras que contienen
    más símbolos $b$ que símbolos $c$ y la que genera las palabras que
    contienen más símbolos $c$ que símbolos $b$.
    \[P_b=\{S_b\rightarrow aS_b|bY|cXX, \ X \rightarrow YX|bY, \ Y
      \rightarrow aY|bY|cX|\varepsilon\}\] Se puede generar cualquier
    secuencia de izquierda a derecha, pero cada vez que se añade una
    $c$ aparecen $X$ que sólo desaparecen añadiendo una $b$. Del
    mismo modo definimos
    \[P_c=\{S_c\rightarrow aS_c|bX'X'|cY', \ X' \rightarrow Y'X'|cY',
      \ Y' \rightarrow aY'|bX'|cY'|\varepsilon\}\] Cualquier palabra
    generada por estas reglas tendrá más símbolos $b$ que símbolos $c$
    o viceversa, pero siempre tendrá distinto número de símbolos $b$ y
    $c$. Por tanto, sólo tenemos que unir ambas gramáticas:
    $P=\{S\rightarrow S_b|S_c\} \cup P_b \cup P_c$
  \end{itemize}
\end{exercise}

\end{document}
