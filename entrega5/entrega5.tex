\documentclass[12pt,spanish]{article}
\pagenumbering{gobble}
% aprovechamiento de la p\'agina -- fill an A4 (210mm x 297mm) page
% Note: 1 inch = 25.4 mm = 72.27 pt
% 1 pt = 3.5 mm (approx)

% vertical page layout -- one inch margin top and bottom
\topmargin      -15 mm   % top margin less 1 inch
\headheight       0 mm   % height of box containing the head
\headsep          0 mm   % space between the head and the body of the page
\textheight     255 mm   % the height of text on the page
\footskip         7 mm   % distance from bottom of body to bottom of foot

% horizontal page layout -- one inch margin each side
\oddsidemargin    0 mm     % inner margin less one inch on odd pages
\evensidemargin   0 mm     % inner margin less one inch on even pages
\textwidth      159 mm     % normal width of text on page

\setlength{\parindent}{0pt}

\usepackage{tikz}
\usetikzlibrary{automata,positioning}

\usepackage[doument]{ragged2e}
\usepackage{babel}
\usepackage[utf8]{inputenc}
\usepackage{amsmath,amsthm,mathtools}
\usepackage{amsfonts,amssymb,latexsym}
\usepackage{enumerate}
\usepackage{subfigure, float, graphicx, caption}
\captionsetup[table]{labelformat=empty}
\captionsetup[figure]{labelformat=empty}
\definecolor{RojoAnayelRey}{rgb}{1,.25,.25}
\usepackage[bookmarks=true,
            bookmarksnumbered=false, % true means bookmarks in 
                                     % left window are numbered         
            bookmarksopen=false,     % true means only level 1
                                     % are displayed.
            colorlinks=true,
            linkcolor=webred]{hyperref}
\definecolor{webgreen}{rgb}{0, 0.5, 0} % less intense green
\definecolor{webblue}{rgb}{0, 0, 0.5}  % less intense blue
\definecolor{webred}{rgb}{0.5, 0, 0}   % less intense red
\definecolor{dkgreen}{rgb}{0,0.6,0}
\definecolor{gray}{rgb}{0.5,0.5,0.5}
\definecolor{mauve}{rgb}{0.58,0,0.82}
\definecolor{MistyRose}{RGB}{255,228,225}
\definecolor{LightCyan}{RGB}{224,255,255}

\usepackage{multicol}
% \usepackage{beton}
% \usepackage[T1]{fontenc}

% Theorem environments

%% \theoremstyle{plain} %% This is the default
\newtheorem{theorem}{Teorema}[section]
\newtheorem{corollary}[theorem]{Corolario}
\newtheorem{lemma}[theorem]{Lema}
\newtheorem{proposition}[theorem]{Proposici\'on}
%\newtheorem{ax}{Axioma}

\theoremstyle{definition}
\newtheorem{definition}{Definici\'on}[section]
\newtheorem{algorithm}{\textrm{\bf Algoritmo}}[section]

\theoremstyle{remark}
\newtheorem{remark}{Observaci\'on}[section]
\newtheorem{example}{Ejemplo}[section]
\newtheorem{exercise}{\textbf{Ejercicio}}%[section]
%\newenvironment{solution}{\begin{proof}[Solution]}{\end{proof}}
\newenvironment{solution}{\begin{proof}[Solución]}{\end{proof}}
\newtheorem*{notation}{Notaci\'on}

%\numberwithin{equation}{section}

%\newcommand{\regla}[2]{
%\begin{array}{c}
%#1\\
%\hline
%#2\\
%\end{array}
%}
\begin{document}

\title{\vspace{-10mm} Modelos de Computación: \\ Relación de problemas
  6} \author{David Cabezas Berrido} \date{\vspace{-5mm}}
\maketitle

\setcounter{exercise}{12}

\begin{exercise}~ Encuentra una gramática libre de contexto en forma
  normal de Chomsky que genere siguiente lenguaje:
  \[L=\{ucv\ : \ u,v \in \{0,1\}^+ \text{ y nº subcadenas `01' en u =
      nº subcadenas `10' en v}\}\]
  Comprueba con el algoritmo CYK si la cadena $010c101$ pertenece al
  lenguaje generado por la gramática.
  
  \vspace{-7mm}

  \begin{multicols}{3}

    \begin{align*}
      S&\rightarrow XcY \\
      S&\rightarrow AHB \\
      H&\rightarrow 01ATB10 \\
      T&\rightarrow H \\
      T&\rightarrow c \\
      ~
    \end{align*}

    \columnbreak

    \begin{align*}
      X&\rightarrow 1A \\
      X&\rightarrow 0A' \\
      A&\rightarrow 1A \\
      A&\rightarrow A' \\
      A'&\rightarrow 0A' \\
      A'&\rightarrow \varepsilon
    \end{align*}

    \columnbreak

    \begin{align*}
      Y&\rightarrow 0B \\
      Y&\rightarrow 1B' \\
      B&\rightarrow 0B \\
      B&\rightarrow B' \\
      B'&\rightarrow 1B' \\
      B'&\rightarrow \varepsilon
    \end{align*}
  \end{multicols}

  Una gramática en forma normal de Chomsky equivalente es: \vspace{-7mm}
  
  \begin{multicols}{3}
    \begin{align*}
      S&\rightarrow D_{Xc}Y \\
      S&\rightarrow D_{AH}B \\
      S&\rightarrow AH \\
      S&\rightarrow HB \\
      S&\rightarrow E_{01AT}E_{B10} \\
      S&\rightarrow E_{01AT}D_{10} \\
      S&\rightarrow E_{01TB}D_{10} \\
      S&\rightarrow E_{01T}D_{10} \\
      ~ \\
      H&\rightarrow E_{01AT}E_{B10} \\
      H&\rightarrow E_{01AT}D_{10} \\
      H&\rightarrow E_{01TB}D_{10} \\
      H&\rightarrow E_{01T}D_{10} \\
      ~ \\
      T&\rightarrow E_{01AT}E_{B10} \\
      T&\rightarrow E_{01AT}D_{10} \\
      T&\rightarrow E_{01TB}D_{10} \\
      T&\rightarrow E_{01T}D_{10} \\
      T&\rightarrow c \\
      ~ \\ ~ \\ ~ \\ ~ \\ ~
    \end{align*}

    \columnbreak

    \begin{align*}
      X&\rightarrow C_1A \\
      X&\rightarrow C_0A' \\
      X&\rightarrow 0 \\
      X&\rightarrow 1 \\
      A&\rightarrow C_1A \\
      A&\rightarrow C_0A' \\
      A&\rightarrow 1 \\
      A&\rightarrow 0 \\
      A'&\rightarrow C_0A' \\
      A'&\rightarrow 0 \\
      ~ \\
      Y&\rightarrow C_0B \\
      Y&\rightarrow C_1B' \\
      Y&\rightarrow 1 \\
      Y&\rightarrow 0 \\
      B&\rightarrow C_0B \\
      B&\rightarrow C_1B' \\
      B&\rightarrow 0 \\
      B&\rightarrow 1 \\
      B'&\rightarrow C_1B' \\
      B'&\rightarrow 1 \\
    \end{align*}

    \columnbreak

    \begin{align*}
      D_{Xc}&\rightarrow XC_c \\
      D_{AH}&\rightarrow AH \\
      D_{01}&\rightarrow C_0C_1 \\
      D_{10}&\rightarrow C_1C_0 \\
      D_{AT}&\rightarrow AT \\
      D_{TB}&\rightarrow TB \\
      ~ \\
      E_{01AT}&\rightarrow D_{01}D_{AT} \\
      E_{B10}&\rightarrow BD_{10} \\
      E_{01TB}&\rightarrow D_{01}D_{TB} \\
      E_{01T}&\rightarrow D_{01}T \\
      ~ \\
      C_c&\rightarrow c \\
      C_0&\rightarrow 0 \\
      C_1&\rightarrow 1 \\
      ~ \\ ~ \\ ~ \\ ~ \\ ~ \\ ~ \\ ~
    \end{align*}
  \end{multicols}

  Ahora aplicaré el algoritmo CYK sobre la palabra $010c101$:

  \begin{table}[H]
    \centering
  \begin{tabular}{|c|cccccc} \hline 0 & \multicolumn{1}{c|}{1} &
\multicolumn{1}{c|}{0} & \multicolumn{1}{c|}{c} &
\multicolumn{1}{c|}{1} & \multicolumn{1}{c|}{0} &
\multicolumn{1}{c|}{1} \\ \hline \shortstack{$X,A,A'$ \\ $Y,B,C_0$} &
\multicolumn{1}{c|}{\shortstack{$X,A,Y$ \\ $B,B',C_1$}} &
\multicolumn{1}{c|}{\shortstack{$X,A,A'$ \\ $Y,B,C_0$}} & \multicolumn{1}{c|}{$T,C_c$} &
\multicolumn{1}{c|}{\shortstack{$X,A,Y$ \\ $B,B',C_1$}} &
\multicolumn{1}{c|}{\shortstack{$X,A,A'$ \\ $Y,B,C_0$}} &
\multicolumn{1}{c|}{\shortstack{$X,A,Y$ \\ $B,B',C_1$}} \\ \hline $Y,B,D_{01}$ &
\multicolumn{1}{c|}{$X,A,D_{10}$} &
\multicolumn{1}{c|}{$D_{Xc},D_{AT}$} & \multicolumn{1}{c|}{$D_{TB}$} &
\multicolumn{1}{c|}{$X,A,D_{10}$} & \multicolumn{1}{c|}{$Y,B,D_{01}$}
& \\ \cline{1-6} $E_{B10}$ & \multicolumn{1}{c|}{$D_{Xc},D_{AT}$} &
\multicolumn{1}{c|}{$S$} & \multicolumn{1}{c|}{$\O$} &
\multicolumn{1}{c|}{$\O$} & & \\ \cline{1-5} $E_{01AT}$ &
\multicolumn{1}{c|}{$S$} & \multicolumn{1}{c|}{$\O$} &
\multicolumn{1}{c|}{$\O$} & & & \\ \cline{1-4} $\O$ &
\multicolumn{1}{c|}{$\O$} & \multicolumn{1}{c|}{$\O$} & & & & \\
\cline{1-3} $S,H,T$ & \multicolumn{1}{c|}{$\O$} & & & & & \\
\cline{1-2} $S,D_{TB}$ & & & & & & \\ \cline{1-1}
  \end{tabular}
\end{table}

$S$ aparece en la última casilla, por tanto la palabra es generada.
  
\end{exercise}

~

\setcounter{exercise}{14}

\begin{exercise}~ Encuentra una gramática libre de contexto en forma
  normal de Chomsky que genere los siguientes lenguajes sobre
  $\{a,0,1\}$:
  \[L_1=\{auava\ |\ u,v \in \{0,1\}^+ \text{ y } u^{-1}=v\}\]
  \vspace{-15mm}
  \begin{multicols}{3}
    \begin{align*}
      S&\rightarrow aHa \\
      H&\rightarrow 1T1 \\
      H&\rightarrow 0T0    
    \end{align*}

    \columnbreak

    \begin{align*}
      T&\rightarrow 1T1 \\
      T&\rightarrow 0T0 \\
      T&\rightarrow a    
    \end{align*}

    \columnbreak

  \end{multicols}

  Gramática equivalente en forma normal de Chomsky: \vspace{-5mm}
  \begin{multicols}{4}
    \begin{align*}
      S&\rightarrow D_{aH}C_a \\
      H&\rightarrow D_{1T}C_1 \\
      H&\rightarrow D_{0T}C_0    
    \end{align*}

    \columnbreak

    \begin{align*}
      T&\rightarrow D_{1T}C_1 \\
      T&\rightarrow D_{0T}C_0 \\
      T&\rightarrow a    
    \end{align*}

    \columnbreak

    \begin{align*}
       D_{aH}&\rightarrow C_aH \\
      D_{1T}&\rightarrow C_1T \\
      D_{0T}&\rightarrow C_0T    
    \end{align*}

    \begin{align*}
      C_a&\rightarrow a \\
      C_0&\rightarrow 0 \\
      C_1&\rightarrow 1         
    \end{align*}

  \end{multicols}

  Comprueba con el algoritmo CYK si $a0a0a$ pertenece a $L_1$:

  \begin{table}[H]
    \centering
    \begin{tabular}{|c|cccc} \hline a & \multicolumn{1}{c|}{0} &
\multicolumn{1}{c|}{a} & \multicolumn{1}{c|}{0} &
\multicolumn{1}{c|}{a} \\ \hline $C_a,T$ & \multicolumn{1}{c|}{$C_0$}
& \multicolumn{1}{c|}{$C_a,T$} & \multicolumn{1}{c|}{$C_0$} &
\multicolumn{1}{c|}{$C_a,T$} \\ \hline $\O$ &
\multicolumn{1}{c|}{$D_{0T}$} & \multicolumn{1}{c|}{$\O$} & \multicolumn{1}{c|}{$D_{0T}$} & \\
\cline{1-4} $\O$ & \multicolumn{1}{c|}{$H$} &
\multicolumn{1}{c|}{$\O$} & & \\ \cline{1-3} $D_{aH}$ &
\multicolumn{1}{c|}{$\O$} & & & \\ \cline{1-2} $S$ & & & & \\
\cline{1-1}
    \end{tabular}
  \end{table}

  $S$ aparece en la última casilla, por tanto la palabra pertenece al
  lenguaje.

  \newpage
  
  \[L_2=\{uvu\ |\ u \in \{0,1\}^+ \text{ y } u^{-1}=v\}\vspace{5mm}\]
  
  Este lenguaje no es independiente del contexto, lo probaré con el
  lema de bombeo. Sea $n\in \mathbb{N}$ arbitrario.\\ 

  La palabra $z=0^n1^n1^n0^n0^n1^n$ pertenece a $L_2$ y tiene longitud
  $6n \geq n$. Tomaremos $c_1=c_3=0^n1^n$, $c_2=1^n0^n$, luego
  $z=c_1c_2c_3$. \\

  Para toda descomposición $z=uxvyw$, $\alpha=xvy$ con
  $|\alpha|\leq n$ y $|xy|\geq 1$.

  \begin{itemize}
  \item $\alpha = 0^k1^l \quad n \geq k+l,\ k,l\geq 1$
    \begin{itemize}
    \item Si son de $c_1$, al hacer $ux^2vy^2w=z'=c_1'c_2c_3$ tendré
      $c_1'\neq c_3$, luego $z' \notin L_2$.
    \item Si son de $c_3$, al hacer $ux^2vy^2w=z'=c_1c_2c_3'$ tendré
      $c_1\neq c_3'$, luego $z' \notin L_2$.
    \end{itemize}
  \item $\alpha = 1^k0^l \quad n \geq k+l,\ k,l\geq 1$
    \begin{itemize}
    \item Deben ser de $c_2$, al hacer $ux^2vy^2w=z'=c_1c_2'c_3$ tendré
      $c_1^{-1}\neq c_2'$, luego $z' \notin L_2$.
    \end{itemize}

  \item $\alpha = 1^k \quad n \geq k\geq 1$
    \begin{itemize}
    \item Si alguno es de $c_1$ (aunque otros sean de $c_2$), al hacer
      $ux^2vy^2w=z'=c_1'c_2'c_3$ tendré $c_1'\neq c_3$, luego
      $z' \notin L_2$.
    \item Si alguno es de $c_2$ (aunque otros sean de $c_1$), al hacer
      $ux^2vy^2w=z'=c_1'c_2'c_3$ tendré $c_2'\neq c_3^{-1}$, luego
      $z' \notin L_2$.
    \item Si son de $c_3$, al hacer $ux^2vy^2w=z'=c_1c_2c_3'$ tendré
      $c_1\neq c_3'$, luego $z' \notin L_2$.
    \end{itemize}

    \item $\alpha = 0^k \quad n \geq k\geq 1$
    \begin{itemize}
    \item Si alguno es de $c_2$ (aunque otros sean de $c_3$), al hacer
      $ux^2vy^2w=z'=c_1c_2'c_3'$ tendré $c_1^{-1}\neq c_2$, luego
      $z' \notin L_2$.
    \item Si alguno es de $c_3$ (aunque otros sean de $c_2$), al hacer
      $ux^2vy^2w=z'=c_1c_2'c_3'$ tendré $c_1\neq c_3'$, luego
      $z' \notin L_2$.
    \item Si son de $c_1$, al hacer $ux^2vy^2w=z'=c_1'c_2c_3$ tendré
      $c_1'\neq c_3$, luego $z' \notin L_2$.
    \end{itemize}
  \end{itemize}
\end{exercise}

\newpage

\textbf{Los siguientes dos ejercicios están mal. En un autómata con
  pila, cuando se lee una palabra partiendo de una configuración
  $(q,H)$ hay que tener en cuenta que puede variar el contenido de la
  pila que hay debajo de $H$.}\\

\setcounter{exercise}{20}

\begin{exercise}~ Si $L_1$ y $L_2$ son lenguajes sobre el alfabeto
  $A$, entonces se define el cociente
  $L_1/L_2=\{u\in A^*\ |\ \exists w \in L_2 \text{ tal que } uw \in
  L_1\}$. Demostrar que si $L_1$ es independiente del contexto y $L_2$
  regular, entonces $L_1/L_2$ es independiente del contexto. \\

  Existirá un autómata no determinista con pila que acepte $L_1$ por
  el criterio de estados finales
  \[M=(Q, A, B, \delta, q_0, R, F)\]

  Defino el autómata
  \[M'=(Q\cup\{q_f\}, A, B, \delta', q_0, R, \{q_f\})\]

  Donde $\delta'$ es un extensión de $\delta$, añadiendo transiciones
  a algunas configuraciones
  \[\delta'(q, \varepsilon, H)=\delta(q, \varepsilon, H)\cup\{(q_f,
    H)\}\]
  $\forall q\in Q,\ H \in B \text{ tales que } \exists w \in L_2
  \text{ cumpliendo } \delta^*(q, w, H)\cap F\times B^* \neq \O$ \\

  En otras palabras, $M'$ acepta (por estados finales) únicamente las
  palabras $u$ que al leerse completamente llevan a una configuración
  para la que existe una palabra $w \in L_2$ que lleva esa
  configuración a un estado final de $M$. Es decir, palabras $u$ tales
  que existe $w \in L_2$ cumpliendo $uw\in L_1$.
\end{exercise}

~

\begin{exercise}~ Si $L$ es un lenguaje sobre $\{0,1\}$, sea $SUF(L)$
  el conjunto de los sufijos de palabras de L:
  $SUF(L)=\{u\in\{0,1\}^*\ |\ \exists v \in \{0,1\}^*, \text{ tal que
  } vu \in L\}$. Demostrar que si $L$ es independiente del contexto,
  entonces $SUF(L)$ también es independiente del contexto. \\

  Existirá un autómata no determinista con pila que acepte $L$ por
  el criterio de estados finales
  \[M=(Q, A, B, \delta, q_0, R, F)\]

  Defino el autómata
  \[M'=(Q\cup\{q_0'\}, A, B, \delta', q_0', R, F)\]
  
  Donde $\delta'$ es una extensión de $\delta$, añadiendo las
  siguientes transiciones:
  \[\delta'(q_0', \varepsilon, R)=\{(q, H)\ |\ \exists v \in
    \{0,1\}^*, \text{ tal que } (q, H) \in \delta^*(q_0, v, R)\}\] Las
  palabras $u$ aceptadas por este autómata son las que llegan a un
  estado final partiendo de cualquier configuración accesible desde
  $(q_0, R)$ por medio de una palabra $v\in\{0,1\}^*$. Es decir,
  palabras $u$ tales que existe $v$ cumpliendo $vu\in L$.
\end{exercise}

\end{document}
