\documentclass[10pt,a4paper]{article}
\usepackage[spanish]{babel}
\usepackage[utf8]{inputenc}
\usepackage[table,xcdraw]{xcolor}
\usepackage[left=3cm,right=3cm,top=3cm,bottom=3cm]{geometry}
\usepackage[export]{adjustbox}
\usepackage{graphicx}
\usepackage{subcaption}
\usepackage{amsmath}



\setlength{\parindent}{0cm}

\title{
Implementación de una clase para gramáticas regulares \\
\large Doble Grado Ingeniería Informática y Matemáticas \\ Modelos de computación}

\author{
David Cabezas Berrido\\
Francisco Miguel Castro Macías
}

\date{08/11/2018}

\begin{document}

\maketitle

Esta práctica consiste en, a partir de un código en python, implementar una clase para representar gramáticas regulares (tipo 3). Se detalla a continuación el formato que deben presentar estas gramáticas para la correcta lectura y escritura, así como la forma en que se han implementado las diversas funciones que se pedían.\\

\section{Formato}
Según nos dice el guión de la práctica, las gramáticas de tipo 3 se representarán así:
\begin{itemize}
\item Una línea con las variables. La primera de ellas será la variable inicial.
\item Una línea con los símbolos terminales.
\item Una línea por cada producción. Cada producción será de la forma 'A \texttt{->} $\alpha$'.
\end{itemize}

\section{Clase y funciones implementadas}
Se ha creado la clase \texttt{RegularGrammar}, que tiene como atributos:

\begin{itemize}
\item \texttt{variables}: las variables que hay en la gramática.
\item \texttt{alphabet}: el alfabeto.
\item \texttt{productions}: un map en el que se guardan las producciones. Por cada variable almacenamos todas las producciones que salen de ella junto a la etiqueta.
\begin{gather*}
\texttt{`source': [`labeldestiny', `labeldestiny', ...]}\\
\texttt{source ->\thinspace \thinspace labeldestiny}
\end{gather*}
\end{itemize}
En esta clase se han creado las siguientes funciones: 
\begin{itemize}
\item \texttt{readFromFile(d):} si \texttt{d} es un fichero que contiene una gramática, la lee.
\item \texttt{writeToFile(d):} escribe la gramática en un fichero con el formato especificado. 
\item \texttt{checkLinearIzdaDcha():} comprueba si la gramática que contiene es linear por la izquierda o linear por la derecha. Devuelve un par \texttt{(i, d)} donde \texttt{i=true} si es linear por la izquierda y \texttt{i=false} en otro caso. Análogo para \texttt{d}.
\item \texttt{checkRegular():} devuelve \texttt{true} si la gramática es regular y \texttt{false} en otro caso.
\item \texttt{toAFND():} genera un objeto de la clase \texttt{afd}, en concreto un autómata finito no determinista con transiciones nulas, que acepta el lenguaje representado por la gramática. Como precondición se impone que la gramática sea linear por la derecha.
\item \texttt{fromAFND(aut):} construye la gramática linear por la derecha que genera el lenguaje aceptado por el autómada \texttt{aut}. 
\end{itemize}

Debemos señalar que somos conscientes de que también se puede pasar de gramática a autómata aun si la gramática es linear por la izquierda. Sin embargo, esto conlleva invertir la gramática, obtener el autómata, invertir el autómata y obtener la gramática de este último autómata.

\section{Test y ficheros de prueba}
Se proporciona por último una serie de ficheros de prueba y una función \texttt{testGrammar()} donde se leen las gramáticas de estos ficheros y se ejecutan las funciones implementadas. 
\end{document}